\section{Prueba del Sistema de Información}
\subsection{Pruebas de software}
\subsubsection{Conceptos generales}
La prueba es un proceso destructivo porque tenemos que indagar sobre lo que hicimos para detectar lo que hicimos mal.\\
Es conocido que la corrección de una falla provoca fallas adicionales, en consecuencia, si una falla aparece debemos probar todo el software.\\
La actividad de prueba puede normalmente dividirse en:
\begin{description}
	\item[Verificación] ¿Estamos construyendo el sistema \emph{correctamente}?
	\item[Validación] ¿Estamos construyendo el sistema correcto?
\end{description}
\subsubsection{Tipos de prueba}
\begin{description}
	\item[Test de Operación] Es el más común. El sistema es probado en operación normal.
	\item[Test de Escala Completa] Ejecutamos el sistema \emph{al máximo} con todos los equipos conectados y usados por muchos usuarios ejecutando casos de uso simultaneamente.
	\item[Test de Performance o de Capacidad] El objetivo de esta prueba es medir la capacidad de procesamiento del sistema.
	\item[Test de Sobrecarga] Cumple la función de determinar cómo se comporta el sistema cuando es \emph{sobrecargado}
	\item[Test Negativos] El sistema es sistemática e intencionalmente usado en forma incorrecta.
	\item[Test basados en requerimientos] Estas pruebas son las que pueden mapearse directamente desde la especificación de requerimientos.
	\item[Test Esgonómico] Son muy importantes si el sistema será usado por gente inexperta. Se prueban la consistencia de la interfaz, consistencia entre las interfaces de los distintos casos de uso, si los menús son lógicos y legibles y si se entienden los mensajes de falta.
	\item[Test de documentación de usuario] Se prueba la documentación del sistema.
	\item[Test de aceptación] Es ejecutado por la organización que solicita el sistema.
\end{description}
\subsubsection{Niveles de prueba}
\paragraph{Prueba de Unidad}
Involucra clases, bloques y paquetes de servicio. Consiste en:
\begin{description}
	\item[Prueba de Especificación o Caja Negra] Verifican el comportamiento de la interfaz de la unidad. \emph{Lo que hace} sin importar \emph{cómo}. Es importante no solo ver que se produzca una salida, sino que la misma sea correcta.
	\item[Prueba Estructural o de Caja Blanca] Se verifica si la estructura interna es la correcta.
	\item[Prueba Basada en estados] Prueba la interacción entre las operaciones de una clase, monitoreando los cambios que tienen lugar en los atributos de los objetos.
\end{description}
\paragraph{Pruebas de Integración}
Una vez que las unidades han sido certificadas en las pruebas, estas deberían integrarse en unidades más grandes y finalmente al sistema. El propósito de las pruebas de integración es determinar si las distintas unidades que han sido desarrolladas trabajan apropiadamente, juntas.\\
Las pruebas de integración se hacen probando cada caso de uso, uno a la vez desde dos puntos de vista:
\begin{description}
	\item[Uno interno] Basado en los diagramas de interacción.
	\item[Uno externo] Basado en las descripciones del modelo de requerimientos.
\end{description}
\paragraph{Prueba del Sistema}
Una vez que se han probado todos los casos de uso por separado se probará el sistema completo. Algunos casos de uso son ejecutados en paralelo y el sistema es sometido a diferentes cargas.
\subsection{Prueba en el Proceso Unificado de Desarrollo}
\subsubsection{Objetivo}
\begin{itemize}
	\item Planificar las pruebas necesarias en cada iteración.
	\item Diseñar e implementar las pruebas creanto los casos de prueba que especifican qué probar.
	\item Realizar las diferentes pruebas y manejar los resultados de cada prueba sistemáticamente.
\end{itemize}
\subsubsection{El papel de la prueba en el ciclo de vida del software}
La realización de pruebas se centra en las fases de elaboración, cuando se prueba la línea base ejecutable de la arquitectura y de construcción cuando el grueso del sistema está implementado. Duerante la fase de transición el centro se desplaza hacia la corrección de defectos durante los primeros usos y a las pruebas de regresión.
\subsubsection{Tipos de pruebas según el PDU}
\begin{description}
	\item[Pruebas de instalación] Verifican que el sistema puede ser instalado en la plataforma del cliente y que el sistema funcionará correctamente cuando sea instalado.
	\item[Pruebas de configuración] Verifican que el sistema funciona correctamente en diferentes configuraciones.
	\item[Pruebas negativas] Intentan provocar que el sistema falle para poder así revelar sus debilidades.
	\item[Peuebas de tensión] Identifican problemas con el sistema cuando hay recurso insuficientes.
\end{description}
\subsubsection{Artefactos}
\paragraph{Modelo de pruebas}
Describe principalmente cómo se prueban los componentes ejecutables en el modelo de implementación con pruebas de integración y de sistema.
\paragraph{Caso de prueba}
Especifica una forma de pribar el sistema, incluyendo la entreda o resultado con la que se ha de probar y condiciones. Casos de prueba comunes:
\begin{itemize}
	\item Caso de prueba que especifica cómo probar un caso de uso o un escenario específico de un caso de uso.
	\item Un caso de prueba que especifica cómo probar una realización de caso de uso de diseño o un escenario específico de una realización.
\end{itemize}
\paragraph{Procedimiento de prueba}
Especifica cómo realizar uno a varios casos de prueba o partes de estos.
\paragraph{Componente de prueba}
Automatiza uno o varios procedimientos de prueba o partes de ellos. Se utilizan para probar los componentes en el modelo de implementación.
\paragraph{Plan de prueba}
Describe estratégias, recursos y planificación de la prueba.
\paragraph{Defecto}
Es una anomalía del sistema.
\paragraph{Evaluación de la prueba}
Es una evaluación de los resultados de la prueba.
\subsubsection{Trabajadores}
\paragraph{Diseñador de pruebas}
Es responsable de la integridad del modelo de pruebas, asegurando que este cumple con su propósito. También planean las pruebas y seleccionan y describen los casos de pruebas y los procedimientos de prueba correspondientes que se necesitan.
\paragraph{Ingeniero de componentes}
Son responsables de los componentes de prueba que automatizan algunos de los procedimientos de prueba.
\paragraph{Ingeniero de pruebas de integración}
Son responsables de realizar las pruebas de  integración que se necesitan para cada construcción producida en el flujo de trabajo de la implementación.
\paragraph{Ingeniero de pruebas de sistema}
Es responsable de realizar las pruebas de sistema necesarias sobre una construcción que muestra el resultado de una iteración completa.
\subsubsection{Actividades}
\paragraph{Planificar la prueba}
El propósito de esta actividad es planificar los esfuerzos de prueba en una iteración llevando a cabo las siguientes tareas:
\begin{itemize}
	\item Describir una estratégia de pruebas.
	\item Estimar los requisitos para el esfuerzo de la prueba.
	\item Planificar el esfuerzo de la prueba.
\end{itemize}
\paragraph{Diseñar la prueba}
Los propósitos de esta actividad son:
\begin{itemize}
	\item Identificar y describir los casos de prueba para cada construcción.
	\item Identificar y estructurar los procedimientos de prueba especificando cómo realizar los casos de prueba.
\end{itemize}
\paragraph{Implementar una prueba}
El propósito de esta actividad es automatizar los procedimientos de prueba creando componentes de prueba si esto es posible.
\paragraph{Realizar pruebas de integración}
En esta actividad se realizan las pruebas de integración necesarias para cada una de las construcciones creadas en una iteración y se recopilan los resultados de las pruebas.
\paragraph{Realizar prueba de sistema}
En esta actividad se realizan las pruebas de sistema necesarias en cada iteración y se recopilan sus resultados.
\paragraph{Evaluar la prueba}
El propósito de esta actividad es evaluar los esfuerzos de prueba en una iteración.
