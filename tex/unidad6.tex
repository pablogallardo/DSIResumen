\section{Evolución del software}
\subsection{Procesos de Evolución del software}
Después de que los sistemas han sido desarrollados inevitablemente van a sufrir cambios si se desea que sigan siendo útiles. Una vez que el software comienza a utilizarse, surgen nuevos requerimientos y los requerimientos existentes cambian. Los cambios eb kis negocios a menudo generan nuevos requerimientos para el software existente.\par
Se puede pensar en la ingeniería del software como un modelo en espiral con requerimientos, dieño, implementación y pruebas que se realizan continuamente durante el ciclo de vida del sistema.
\subsection{Dinámica de evolución}
La dinámica de evoluci{on de los programas es el estudio de los cambios del sistema.
\subsubsection{Leyes de Lehman}
\paragraph{Cambio continuado}
Un programa que se usa en un entorno real necesariamente debe cambiar o se volverá progresivamente menos útil en ese entorno.
\paragraph{Complejidad creciente}
A medida que un programa en evolución cambia, su estructura tiende a ser cada vez más compleja.
\paragraph{Evolución prolongada del programa}
La evolución de los programas es un proceso autorregulativo. Los atributos de los sistemas tales como el tamaño, tiempo entre entregas y el número de errores documentados son aproximadamente invariantes para cada entrega.
\paragraph{Estabilidad organizacional}
Durante el tiempo de vida de un programa, su velocidad de desarrollo es aproximadamente constante e independiente de los recursos dedicados al desarrollo del sistema.
\paragraph{Conservación de la familiaridad}
Durante el tiempo de vida de un sistema el cambio incremental en cada entrega es aproximadamente constante.
\paragraph{Crecimiento continuado}
La funcionalidad ofrecida por los sistemas tiene que crecer continuamente para mantener la satisfacción de los usuarios.
\paragraph{Decremento de la calidad}
La calidad de los sistemas comenzará a disminuir a menos que dichos sistemas se adapten a los cambias en su entorno de funcionamiento.
\paragraph{Realimentación del sistema}
Los procesos de evolución incorporan sistemas de realimentación multiagente y multibucle y estos deben ser tratados como sistemas de realimentación para lograr una mejora significativa del producto.
\subsection{Mantenimiento del software}
El mantenimiento del software es el proceso general de cambiar un sistema después de que este ha sido entregado. Existen tres tipos diferentes de mantenimiento de software.
\begin{description}
\item[Mantenimiento para reparar defectos del software] Se reparan errores de código y requerimientos.
\item[Mantenimiento para adaptar el software a diferentes entorno operativos] Este tipo de mantenimiento se requiere cuando cambia algún aspecto del entorno del sistema como por ejemplo el hardware.
\item[Mantenimiento para añadir o modificar las funcionalidades del sistema] Es necesario cuando los requerimientos del sistema cambian como respuesta a cambios organizacionales o del negocio.
\end{description}
\subsubsection{Costes del mantenimiento}
\begin{description}
\item[Estabilidad del equipo] Es normal que el equipo de desarrollo se disuelva y la gente trabaje en nuevos proyectos. Un nuevo equipo de desarrolladores tardará tiempo en comprender el sistema.
\item[Responsabilidad contractual] El contrato de mantenimiento puede darse con una compañía diferente en lugar de con el desarrollador original del sistema. No existe incentivo para escribir un código que sea fácil de mantener.
\item[Habilidades del personal] El personal de mantenimiento a menudo no tiene experiencia y no está familiarizado con el dominio de la aplicación.
\item[Edad y estructura del programa] A medida que pasa el tiempo la estructura del software tiende a degradarse con los cambios por lo que se vuelve más difícil de comprender y modificar.
\end{description}
\subsection{Administración de Sistemas heredados}
El proceso de evolución de los sistemas implica comprender el programa que tiene que cambiarse y a continuación implementar estos cambios. Los sistemas heredados más antiguos son difíciles de comprender y de cambiar.\par
Para simplificar los problemas de cambiar sus sistemas heredados, una compañía puede decidir hacer reingeniería sobre esos sistemas para mejorar su estructura y comprensibilidad. La reingeniería del software se refiere a la reimplementación de los sistemas heredados para hacerlo más mantenibles, puede implicar redocumentar el sistema, organizar y reestructurar el sistema, traducir el sistema a un lenguaje de programación más moderno y modificar y actualizar la estructura y valores de los datos del sistema.\par
Ventajas:
\begin{description}
\item[Riesgo reducido] Pueden cometerse errores en la especificación o puede haber problemaws en el desarrollo. Los retrasos en la introducción del nuevo software pueden significar pérdidas en el negocio e incurrir en costes adicionales.
\item[Coste reducido] El coste de hacer reingeniería es significativamente menor que el coste de desarrollar nuevo software.
\end{description}
\subsubsection{Actividades de el proceso de reingeniería}
\begin{description}
\item[Traducción del código fuente] El programa es traducido desde un lenguaje de programación antiguo a una versión más moderna del mismo lenguaje o a un lenguaje diferente.
\item[Ingeniería inversa] El programa se analiza y se extrae información a partir de él.
\item[Mejora de la estructura de los programas] La estructura de control del programa se analiza y modifica para hacerla más fácil de leer y comprender.
\item[Modularización de los programas] Se agrupan las partes relacionadas del programa y se elimina la redundancia en donde resulta adecuado.
\item[Reingeniería de datos] Los datos procesados por el programa se cambian para reflejar los cambios en el.
\end{description}