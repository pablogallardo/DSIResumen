\section{Despliegue del Sistema de Información}
\subsection{Problemática del Despliegue de software}
Los desarrolladores de software a veces declaran la victoria demasiado pronto. Se olvidan que lo que marca el desarrollo exitoso del sistema es la disposición del cliente a usar el software terminado en vez de simplemente un software que compila satisfactoriamente.
\subsection{Técnicas de despliegue en función de las tecnologías existentes}
\subsubsection{Despliegue del software en sistemas hechos a medida}
Cada sistema de esta clase de software usualmente es único en su género y a veces estan asociados a hardware hecho a medida también. En estos casos el envolvimiento del desarrollador varía según el caso.
\subsubsection{Despliegue del software enlatado y descargable de internet}
El instalador será el usuario final. En este caso el envolvimiento del desarrollador será hacer el sitio web del producto o crear el packaging para distribuir el producto.
\subsection{Proceso de despliegue del producto}
\subsubsection{El rol del despliegue en el Ciclo de Vida del Software}
Las actividades de despliegue se enfocan en la fase de transición y representan la culminación del esfuerzo de los desarrolladores de software. Durante la fase de transición el usuario final puede empezar a usar el producto para ver como funciona en un entorno real de ejecución. Versiones de prueba o beta al final de series de iteraciones de desarrollo dan la oportunidad para sugerencias finales para ajustar el producto.
\subsubsection{Artefactos}
\begin{itemize}
	\item Software ejecutable.
	\item Artefactos de instalación (Scripts, herramientas, guías, etc).
	\item Notas de lanzamiento.
	\item Material de soporte.
	\item Material de entrenamiento.
	\item Lista de materiales.
	\item Lanzamiento Maestro.
	\item Empaquetamiento.
	\item Productos artístico.
	\item Material impreso.
	\item Producto en medio de almacenamiento.
\end{itemize}
\subsubsection{Trabajadore}
\begin{description}
	\item[El Encargado de Despliegue] Planea y organiza el despliegue.
	\item[El Encargado del Proyecto] Es el mediador entre el cliente y el responsable de aprobar el despliegue.
	\item[El escritor técnico] Planea y produce el material para el usuario final.
	\item[Desarrollador de Cursos de Entrenamiento] Planea y produce material de entrenamiento.
	\item[El artísta gráfico] Es responsable de todo el trabajo artístico relacionado con el producto.
	\item[El tester] Es el responsable de asegurar que el producto haya sido probado de forma adecuada.
	\item[El implementador] Crea scripts de instalación y artefactos relacionados que ayudan al usuario final a instalar el producto.
\end{description}
\subsubsection{Actividades}
\paragraph{Planear el despliegue}
No solo se encarga de planear cómo desarrollar los entregables sino también asegurar que el usuario tenga la información necesaria para usar el nuevo software.
\paragraph{Desarrollar material de soporte}
Desarrollar la información que va a necesaritar el usuario final para instalar, operar, usar y mantener el producto.
\paragraph{Probar el producto en sala de desarrollo}
Esto ayuda a determinar si el producto es suficientemente maduro para el lanzamiento beta.
\paragraph{Crear el lanzamiento}
El propósito de esta actividad es asegurarse que el producto esta listo para ser entregado al cliente.
\paragraph{Pruebas beta de lanzamiento}
El propósito de esta actividad es hacer que el software entregado sea instalado por el usuario final para que de feedback de su rendimiento y usabilidad.
\paragraph{Probar el producto en lugar de instalación}
Consiste en que el cliente instale el producto en el lugar donde se usará habitualmente y sea probado ahí.
\paragraph{Empaquetar el producto}
Consisten en actividades de empaquetamiento del software que será distribuido.
\paragraph{Proveer acceso al link de descarga del producto}
Hacer el producto accesible desde internet para navegadores y sitios web.
