\section{Unidad 3: Diseño de Arquitecturas de Software}
\subsection{Diseño Arquitectónico}
\subsubsection{¿Qué es?}
Es la asignación de modelos de requerimiento esenciales a una tecnología específica. Es el plan del Diseño
\subsubsection{Rol del arquitecto}
\begin{itemize}
\item Revisa y negocia los requerimientos
\item Documentar la Arquitectura
\item Comunica la arquitectura, asegurándose que los involucrados comprendan.
\item Direcciona requerimientos no funcionales a la arquitectura
\item Configura la arquitectura de hardware
\item Se asegura que la arquitectura es respetada
\item Trabaja con el Administrador del Proyecto, ayudando en la planificación, la estimación la distribución de tareas y la calendarización del proyecto.
\end{itemize}
\subsubsection{Ventajas de diseñar y documentar la arquitectura}
\begin{itemize}
	\item Comunicación con los involucrados.
	\item Análisis del sistema.
	\item Reutilización a gran escala.
\end{itemize}
\subsubsection{Requerimiento no funcionales afectados por la arquitectura}
\begin{description}
	\item[Rendimiento] Si es un requerimiento crítico, la arquitectura debería diseñarse para identificar las operaciones críticas en un pequeño número de subsistemas con la mínima comunicación como sea posible entre ellas.
	\item[Protección] Si es un requerimiento crítico, debería usarse una arquitectura estructurada en capas, con las más crñiticas protegidas en capas más internas y aplicando una validación de seguridad de alto nivel en dichas capas.
	\item[Seguridad] Si es un requerimiento crítico, la arquitectura debería diseñarse para que las operaciones relacionadas con la seguridad se localizaran en un único subsistema o en un pequeño número de subsistemas.
	\item[Disponibilidad] Si es un requerimiento crítico, la arquitectura debería diseñarse para incluir componentes redundantes y para que sea posible reemplazar y actualizar componentes sin detener el sistema.
	\item[Mantenibilidad] Si es un requerimiento crítico, la arquitectura del sistema debería diseñarse usando componentes independientes de grano fino que puedan modificarse con facilidad.
\end{description}
\subsubsection{Conflictos arquitectónicos en requerimientos no funcionales}
\begin{itemize}
	\item Utilizar componentes de granularidad alta mejora la performance pero reduce mantenibilidad.
	\item La introducción de datos redundantes mejora la disponibilidad peor hace la seguridad más difícil.
	\item La localización de aspectos de seguridad relacionados usualmente significa más comunicación por lo tanto degrada la performance.
\end{itemize}
\subsubsection{Modelado de la Arquitectura}
El modelo arquitectónico mapea los requerimientos funcionales del análisis a una arquitectura tecnológica. Debe considerar información sobre:
\begin{itemize}
	\item Volúmenes de Datos.
	\item Funcionalidad más demandada en el negocio.
	\item Distribución geográfica.
	\item Distribución del Procesamiento de datos.
	\item Dónde se guardarán los datos.
	\item Cuáles procesos se ejecutarán en qué procesadores y que tanta comunicación se requerirá entre ellos.
\end{itemize}
\subsection{Patrones Arquitectónicos}
\subsection{Vistas Arquitectónicas}
